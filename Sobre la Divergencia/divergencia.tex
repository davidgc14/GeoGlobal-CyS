\documentclass[fleqn]{article}

%\pgfplotsset{compat=1.17}

\usepackage{mathexam}
\usepackage{amsmath}
\usepackage{amsfonts}
\usepackage{graphicx}
\usepackage{subcaption}
\usepackage{systeme}
\usepackage{microtype}
\usepackage{multirow}
\usepackage{pgfplots}
\usepackage{listings}
\usepackage{tikz}
\usepackage{dsfont} %Numeros reales, naturales...
\usepackage{cancel}
\usepackage{hyperref}


%\graphicspath{{images/}}
\newcommand*{\QED}{\hfill\ensuremath{\square}}



\author{David García Curbelo}
\title{CyS}

\pagestyle{empty}


\def\R{\mathds{R}}
\def\Z{\mathds{Z}}
\def\N{\mathds{N}}
\def\X{\mathbf{X}}

\def\sup{$^2$}

\def\next{\quad \Rightarrow \quad}

\begin{document}
    \setcounter{page}{1}
    \pagestyle{plain}

    \begin{center}
        {\LARGE\bf{Sobre el concepto de Divergencia}} \\
        {\large\bf{Geometría Global de Curvas y Superficies}}\\
    \end{center}

    Para considerar de manera correcta la interpretación de divergencia, consideremos primero un campo vectorial, e imaginemos dicho campo como 
    si representara el movimiento de un fluido (toda la explicación que precede se desarrollará en $\R^2$, y de forma análoga se puede extrapolar a cualquier
    dimensión finita $\R^n, \thinspace n \in \N$). 

    Consideremos por $\X$ a nuestro campo vectorial, es decir, a una aplicación dada por 
    $$\X : \R^2 \longrightarrow \R^2$$
    tal que asocia a cada punto un vector, representando así el movimiento de nuestra partícula por el fluido. Un ejemplo para nuestro campo puede ser 
    $$\X (x,y) = \left[
        \begin{matrix}
        \X_1(x,y)\\
        \X_2(x,y)
    \end{matrix} \right]
    = \left[
        \begin{matrix}
        2x-y\\
        y^2
    \end{matrix} \right]
    $$
    Donde los valores de entrada de $\X$ son como mencionamos los puntos de un espacio de dos dimensiones, $(x, y)$, y los valores de salida son vectores en 
    dos dimensiones, los cuales están asignados a un correspondiente punto $(x,y)$ en el campo vectorial.

    Una buena manera de imaginarse los campos vectoriales es, como ya comentamos, imaginando el movimiento de un fluido que podrían representar. Específicamente,
    para cada punto $(x,y)$ en un espacio de dos dimensiones, imaginemos una partícula en el punto $(x,y)$ moviéndose en la dirección del vector asignado a ese 
    punto: $\X (x,y)$. Más aún, podemos suponer el módulo de este vector (su longitud en el campo vectorial) como la rapidez del movimiento de nuestra partícula
    en el campo. (Ver Figura 1)

    La curiosidad de estos campos es que, mientras el fluido se mueve, algunas regiones tienden a ser más o menos densas, dependiendo de las características del
    campo (hacia dónde apunten los vectores). Si por casualidad, en una zona del campo todos los vectores apuntan a nuestro punto $(x,y)$, éste con el tiempo se 
    volverá más denso, pues en un entorno cercano cada vez tendrá más y más puntos. Y es aquí donde surge la duda: ¿De qué manera podemos medir el cambio en la
    densidad de las partículas alrededor de un punto $(x,y)$ mientras estas se mueven a lo largo de lso vectores dados por $\X (x,y)$?

    Y es aquí donde toma papel el concepto que veníamos buscando: la {\bf{divergencia}}. La divergencia no es otra cosa que un operador, en el cual tenemos en cada 
    componente las parciales respecto a cada una de nuestras variables. Ésta se suele denotar por el símbolo de gradiente, $\nabla$, de la siguiente manera

    $$\nabla = \left[
        \begin{matrix}
        \frac{\partial}{\partial x}\\
        \frac{\partial}{\partial y}
    \end{matrix} \right]$$

    Recuerdo que todas estas definiciones se pueden extrapolar a cualquier dimensión. Considerando ahora la notación del producto, podemos obtener bastante directo
    lo que andábamos buscando, una función que nos indique la densidad de nuestro campo: la divergencia de nuestro campo. "Multiplicando" el operador $\nabla$ y 
    nuestro campo, obtenemos la divergencia de nuestro campo vectorial

    $$
    \nabla * \X = div(\X) = \left[
        \begin{matrix}
        \frac{\partial}{\partial x}\\
        \frac{\partial}{\partial y}
    \end{matrix} \right]
    \left[ \X_1, \X_2 \right] = \frac{\partial \X_1}{\partial x} + \frac{\partial \X_2}{\partial y}
    $$

    Esta notación que hemos usado aquí no es del todo rigurosa, pero da a entender de una manera intuitiva cómo se comporta la divergencia del campo. Retomemos el
    ejemplo que consideramos al principio de este escrito
    $$\X (x,y) = \left[
        \begin{matrix}
        2x-y\\
        y^2
    \end{matrix} \right]
    $$
    Su divergencia vendría dada por $div(\X) = \frac{\partial}{\partial x}(2x-y) + \frac{\partial}{\partial y}(y^2) = 2+2y$. Más adelante veremos cómo podemos sacar 
    conclusiones de hacia dónde evoluciona el campo y, sobre todo, si tiende a ser denso y dónde. Otra manera de entender la fórmula de la divergencia (que a lo mejor
    te resulta útil) es considerando la divergencia como la traza del jacobiano de nuestro campo vectorial.

    Ahora, supongamos que ya tenemos nuestro campo sobre la mesa y en nuestro lapicero nuestra función divergencia del campo. Tomemos un punto cualquiera de nuestro campo, 
    digamos por ejemplo el punto $(x_0, y_0)$ y evaluémoslo en nuestra divergencia$div(\X) (x, y)$:
    
    \begin{enumerate}
        \item[] $\quad * \quad div(\X) (x_0, y_0) < 0$. Si la divergencia es negativa en dicho punto, quiere decir que el fluido que se mueve por el campo tenderá a hacerse {\bf{más denso}}
                en el punto $(x_0, y_0)$. (Ver Figura 2)
        \item[] $\quad * \quad div(\X) (x_0, y_0) > 0$. De manera análoga al caso anterior, el fluido de partículas que domina en nuestro campo tenderá a dispersarse: hacerse 
                {\bf{menos denso}} en dicho punto. (Ver Figura 3)
        \item[] $\quad * \quad div(\X) (x_0, y_0) = 0$. Finalmente tenemos el caso nulo (muy importante en mecánica de fluidos y electrodinámica). Esto indica que, aunque el 
                fluido se mueve, {\bf{su densidad permanece constante}} a lo largo del tiempo. 
    \end{enumerate}

    Si consideramos nuestro ejemplo anterior, vemos que la divergencia $div(\X) = 2+2y$ de nuestro campo nos indica que para todo $y>-1$ y cualquiera que sea $x$ en la coordenada
    de los puntos de nuestro campo, nuestro campo se volverá menos denso, mientras que en el otro semiplano $\left\{(x,y) \thinspace / \thinspace y<-1, \thinspace (x,y) \in \R^2\right\}$ 
    el campo se vuelve más y más denso a cada instante. A su vez, en la recta $y=-1$ vemos que la densidad del campo se mantiene constante.

    Para poder ver de manera más completa de los ejemplos de divergencia, dejo un alcance de todos unos vídeos sobre divergencia de distintos campos, haciendo click aquí:
    \url{https://www.youtube.com/watch?v=EpXU3xyVAzo&list=PLKXWoWb0qgQ86VRU20ITYcnaNDJU5EnDn&index=2&ab_channel=3Blue1BrownClips}
    
    \newpage
    \begin{figure}
        \centering
        \begin{subfigure}[b]{0.45\linewidth}
            \includegraphics[width=\linewidth]{campos/campo (1).png}
            \caption{Instante cero}
            \label{fig:campo_divergente}
        \end{subfigure}
        \begin{subfigure}[b]{0.45\linewidth}
            \includegraphics[width=\linewidth]{campos/campo (2).png}
            \caption{Instante uno}
            \label{fig:campo_divergente}
        \end{subfigure}
        \caption{Ejemplo de campo}
        \label{fig:campo}
    \end{figure}

    \begin{figure}
        \centering
        \begin{subfigure}[b]{0.45\linewidth}
            \includegraphics[width=\linewidth]{campos/campo (5).png}
            \caption{Instante cero}
            \label{fig:campo_divergente}
        \end{subfigure}
        \begin{subfigure}[b]{0.45\linewidth}
            \includegraphics[width=\linewidth]{campos/campo (6).png}
            \caption{Instante uno}
            \label{fig:campo_divergente}
        \end{subfigure}
        \caption{Evolución a un campo más denso}
        \label{fig:campo}
    \end{figure}

    \begin{figure}
        \centering
        \begin{subfigure}[b]{0.45\linewidth}
            \includegraphics[width=\linewidth]{campos/campo (3).png}
            \caption{Instante cero}
            \label{fig:campo_divergente}
        \end{subfigure}
        \begin{subfigure}[b]{0.45\linewidth}
            \includegraphics[width=\linewidth]{campos/campo (4).png}
            \caption{Instante uno}
            \label{fig:campo_divergente}
        \end{subfigure}
        \caption{Evolución a un campo menos denso}
        \label{fig:campo}
    \end{figure}


\end{document}