\documentclass[fleqn]{article}

%\pgfplotsset{compat=1.17}

\usepackage{mathexam}
\usepackage{amsmath}
\usepackage{amsfonts}
\usepackage{graphicx}
\usepackage{subcaption}
\usepackage{systeme}
\usepackage{microtype}
\usepackage{multirow}
\usepackage{pgfplots}
\usepackage{listings}
\usepackage{tikz}
\usepackage{verbatim} %comentarios de párrafos
\usepackage{dsfont} %Numeros reales, naturales...
\usepackage{cancel}
\usepackage{hyperref}
\usepackage[spanish]{babel}
\usepackage[utf8]{inputenc} %Indice


%\graphicspath{{images/}}
\newcommand*{\QED}{\hfill\ensuremath{\square}}


%Estructura de ecuaciones
%\setlength{\textwidth}{15cm} \setlength{\oddsidemargin}{5mm}
%\setlength{\textheight}{23cm} \setlength{\topmargin}{-1cm}


\author{David García Curbelo}
%\title{}

\pagestyle{empty}


\def\R{\mathds{R}}
\def\Z{\mathds{Z}}
\def\N{\mathds{N}}
\def\X{\mathbf{X}}
\def\S{\mathbb{S}}

\def\Jac{\text{Jac}}

\def\next{\quad \Rightarrow \quad}

\def\s{\thinspace}

\def\ss{\thinspace \thinspace}

\begin{document}
    \setcounter{page}{1}
    \pagestyle{plain}

    \begin{center}
        {\LARGE\bf{Examen Final Resuelto}} \\
        {\large\bf{Geometría Global de Curvas y Superficies}}\\
    \end{center}

    \textbf{Ejercicio 5. }\textit{Sea $A$ una matriz cuadrada de oren tres regular. Demostrar que la aplicación 
        $ \phi : \S^2 \rightarrow \S^2 $ dada por 
        $$\phi (p) = \frac{Ap}{|Ap|}$$
        es un difeomorfismo. Usar la fórmula del cambio de variables para comprobar que 
        $$\int_{\S^2} \frac{1}{|Ap|^3} dp = \frac{4\pi}{|\det A|}.$$
        }

        La aplicación $\phi$ es claramente diferenciable ya que es una restricción de una aplicación definida en $\R^3 - \{0\}$ restringida
        a $\S^2$. Como la matriz es cuadrada y regular, entonces tiene inversa y podemos definir dicha aplicación como 
        $\varphi :\S^2 \rightarrow \S^2$ tal que
        $$\varphi(p) = \frac{A^{-1}p}{|A^{-1}p|}, \quad \quad \forall p \in \S^2$$
        la cual es la inversa de nuestra función $\phi$, y por el mismo razonamiento que antes vemos que es diferenciable su inversa, y por tanto 
        $\phi$ es un difeomorfismo. Para terminar, calculemos el valor absoluto del Jacobiano de $\phi$ como sigue. Para ello primero necesitamos
        $$(d\phi)_p (v) = \frac{Av}{|Ap|} - \frac{\langle Ap, Av \rangle}{|Ap|^3} Ap, \quad \quad \forall v \in T_p\S^2 \s \forall p \in \S^2$$
        Considerando ahora una base ortonormal de $\R^3$ con $\{e_1, e_2, p\}$, con $e_1, e_2 \in T_p\S^2 \s p \in \S^2$ tenemos el Jacobiano:
        $$ | \Jac \phi| (p) = \frac{1}{|Ap|^3} |det(Ae_1, Ae_2, Ap)| = \frac{|\det A|}{|Ap|^3}$$
        Y por tanto tenemos
        $$\int_{\S^2} | \Jac \phi| (p) = \int_{\S^2} \frac{|\det A|}{|Ap|^3} \next \int_{\S^2} \frac{1}{|Ap|^3} = \int_{\S^2} \frac{| \Jac \phi| (p)}{|\det A|} = \frac{4\pi}{|\det A|}.$$

        \newpage

    \textbf{Ejercicio 6. }\textit{Sea $S$ una superficie compacta con curvatura de Gauss no nula en cada punto y tal que su aplicación de Gauss
        es inyectiva. Probar que
        $$\int_S K = 4\pi.$$
        }

        Notemos en una primera instancia que, como $K \neq 0 \forall p \in S$, no puede haber cambio de signo en dicha curvatura, y como sabemos
        que la curvatura de Gauss no puede ser siempre no positiva, tenemos que $K>0$. Tomemos ahora la aplicación de Gauss dada por $N: S \rightarrow \S^2$,
        la cual vemos fácil que es un difeomorfismo local, ya que $K(p)\neq 0$ y $K(p) = \det(dN)_p$, entonces particularmente $\det(dN)_p \neq 0$. Pero además,
        nuestra aplicación de Gauss lleva abiertos en abiertos, y por tanto $N(S) \subseteq \S^2$ tiene que ser un subconjunto abierto, pero vemos que como $S$
        es compacto, también tiene que ser cerrado. Por ello, concluimos que $N$ es un difeomorfismo global. Aplicando ahora la fórmula del cambio de variable tenemos
        $$\int_S |K| = \int_S K = \int_{\S^2} |\Jac N| = A(\S^2) = 4\pi.$$
\end{document}